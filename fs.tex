
操作系统内核是一个抽象层,提供统一的接口给应用程序。网络是套接字(socket),对于存储设备,就是文件系统。
UNIX类操作系统使用树形文件系统结构。文件被组织成目录,目录和文件还可以组织成父级目,一直到根目录。

TODO 树形目录插图。 %TODO

文件系统提供了统一的方式组织数据。应用程序和用户不需要关心数据存储在哪个磁盘扇区,只需要使用一个文件名去引用即可。
在内核,同样是提供“文件”这么一个抽象的接口,不同的人提出了不同的数据组织方式。文件系统类型就是文件在磁盘上的具体组织格式。
按照数据存储设备的类型,有块设备文件系统,虚拟文件系统和网络文件系统三大类。硬盘是最主要的块设备,多数块设备文件系统也都是为磁盘设计的。

\section{磁盘文件系统}

磁盘是使用磁性碟片作为数据存储介质的随机访问存储设备。按照碟片材料可以分硬磁盘和软磁盘。软磁盘已经被淘汰,现在说的磁盘主要就是硬磁盘。也就是硬盘。
在硬盘上存储文件,是内核设计文件系统设计的主要目的。

\subsection{硬盘结构}

TODO 硬盘图片 %TODO

硬盘具备随机访问能力,它不像磁带那样,需要经历漫长的“进带”“倒带”操作才能定位到莫个位置进行访问。
硬盘的基本结构是碟片和磁头,磁头能沿着碟片的半径移动,随着碟片的旋转和磁头的移动,碟片上的任一位置都能被磁头快速定位。

TODO 磁头和碟片的示意图 %TODO

为了最大化的利用磁碟的可用空间,硬盘为每个碟片配两个磁头分别用于正反面读写。为了提高容量,除了加大存储密度外,还可以采取“堆砌”——使用多片磁碟。为了简化实现,降低成本,同一时刻只能有一个磁头能和控制电路通讯;所有的磁头是统一定位的——同进同退——因为这样只需要一个磁头定位电机\footnote{移动磁头臂的电机类似喇叭里头的音圈,又称音圈电机}。磁头臂步进一次,每个磁头都能访问磁碟上狭窄的一圆周形数据区。这个数据区叫磁道。所有同半径的磁道构成一个柱面。磁头臂从最外侧移动到最内侧,步进的次数就是柱面数。每个磁道都被分隔为长度一致的扇区。
显然硬盘容量就是柱面数\texttimes磁头数\texttimes每磁道扇区数\texttimes每扇区容量 \footnote{如今的硬盘,每个柱面包含的扇区数并不相等。这个简单的公式,也只适用于老式硬盘。现在的硬盘,容量需要每个柱面的扇区数一个一个加上来。}。
柱面的编号是从外侧开始的,最外侧是柱面0。图 \ref{fig:Cylinder_Head_Sector} 是一个4磁碟8磁头的扇区和柱面示意图,该硬盘使用的还是所有磁道等扇区数的形式。

\chatu[width=4.2cm]{Cylinder_Head_Sector}{磁头和柱面和扇区}

硬盘访问数据的时候,首先移动磁头臂,定位到目的数据所在柱面;接着选中对应磁面的磁头,开始等待目的扇区随着磁碟的旋转移动到磁头下;扇区到达磁头位置的时候,记录磁头读取出的电信号,转化为二进制数据即可向主机反回该扇区的数据了。写数据也是一样,先定位,然后用待写入数据控制磁头,磁头将扇区磁化,数据就永久保存到磁盘上了。
由此可见一个扇区的地址可以用 “柱面X磁头Y扇区Z”(CHS) \footnote{扇区的编号从1开始,磁头和柱面编号从0开始。}来描述\footnote{那是老式硬盘的做法了,现在都使用LBA(Logical Block Addressing)寻址了。LBA寻址为每个扇区分配一个逻辑扇区号(从0开始),由硬盘自己将逻辑扇区号映射到“柱面+磁头+扇区”的形式。逻辑扇区号采取线性编排措施,首先编排柱面0磁头0的扇区,然后是柱面0磁头1的扇区...... 逻辑扇区向操作系统屏蔽了磁盘的物理结构。操作系统在访问磁盘的时候,可以选择使用CHS格式的地址,也可以使用LBA形式。因为CHS格式的地址因为历史原因不能访问超过8.3G边界的扇区,所以大于8.3G的硬盘必须使用LBA寻址。}。


音圈电机移动磁头臂的速度,决定了硬盘执行柱面定位的速度。所花时间被称为寻道时间。因为这个是机械运动,受限于磁头的惯性,不可能无限提高。磁头移动到目标柱面位置后,等待目标扇区移动到磁头下,所花的时间为等待时间。显然磁盘旋转速度越高,等待时间越短。每次访问扇区的时候,寻道和等待的时间都依赖目的扇区的位置和当前磁头的位置,并不是固定不变的。所以硬盘参数上标明的是平均寻道时间和平均等待时间。
因为硬盘以非常高的恒定速度旋转\footnote{民用桌面硬盘通常转速为7200RPM(每分钟7200转)。服务器使用的硬盘转速普遍超过10000RPM,当然,转速高了噪音也大,所以桌面硬盘少有大于7200RPM的。}
所以硬盘访问时间主要受制于寻道时间。操作系统在读写硬盘的时候,都要尽力减少磁头移动的次数,减少花在寻道上的时间。


磁盘文件系统的设计目标就要考虑到磁盘在数据存储上的特点:
\begin{itemize}
\item 数据可以随机访问。
\item 每次定位需要花费不小的寻道时间。
\item 数据相离越近,寻道时间越短。
\item 最小访问单位是扇区。扇区的典型大小是512字节,近年来也有4K字节扇区的硬盘出现在市场上。
\end{itemize}

\subsection{非日志文件系统}

针对磁盘升级的各个文件系统或多或少有点相同点:
\begin{itemize}
\item 在分区的第一个扇区保存文件系统的元信息。这个保存元信息的扇区又称超级块,SuperBlock。超级块损害将导致内核无法识别分区所使用的文件系统。
\item 文件系统将磁盘空间划分成两大类,目录区和数据区。目录区用来存储文件系统的目录结构,存储文件名,文件权限等信息。文件的具体内容保存到数据区。
\item 数据区使用“簇”来组织。簇是文件系统的最小存储单位。一个文件总是占用一个或者多个簇。簇的大小在格式化的时候确定。簇的大小必须是2的幂。通常使用512字节,1024字节,2K和4K字节大小的簇。
因为再小的文件都要占用一个簇,并且文件大小不是簇大小的整数倍的时候,最后一块簇会有多余的空间被浪费,因此更小的簇有助于节约磁盘空间。但是更大的簇使得一个文件使用更少数目的簇保存,有利于减少磁盘碎片并提高性能。
\item 大小超过一个簇的文件,使用“链式”的办法连接多个簇。所以单一文件并不总是保存在连续的簇上的,这就带来了文件碎片,使得读取文件的时候频繁寻道,降低性能。文件系统的一个设计目标就是尽量减少文件碎片。
\end{itemize}

操作系统在写入文件到磁盘的时候,会出现中途被打断的情况\footnote{这种中断的结果是文件系统被卸载,不过由于不是使用umount命令进行卸载,所以叫非正常卸载。},不论突然中断的原因是是磁盘被暴力拔出,还是突然停电,或者是死机,都会导致文件系统出现不一致。

轻则文件丢失:可能是文件被内核缓存在内核中,还没有写入硬盘;重者导致文件系统不能使用:文件正在写入硬盘,并且操作系统修改目录区数据,让文件内容以指向新的簇,突然断电导致目录区数据被破坏。

如果出现了非正常卸载,下次文件在挂载之前就需要执行文件系统检查。这个检查由/sbin/fsck执行。fsck检查文件系统是否出现问题,有的话自动修复。fsck是一个非常耗时的操作。它需要扫描文件系统上的所有文件,所有目录结构,才能确定那些数据是损坏的。为了应对这个,发明了日志文件系统,如 ext2 的升级版 ext3和ext4。

\subsection{日志文件系统}

非日志文件系统在非正常卸载后需要花很久的时间进行扫描,实际上在经历非正常卸载前,内核通常来说只对其中很少数量的文件进行操作,所以理论上只要重新扫描这些文件即可。这就大大加快了磁盘检查的速度。
完成这一个办法就是采取“日志”措施:对文件系统进行修改前,都往特殊区域写日志先。如果在写日志的时候断电,那么不完整的日志被忽略。如果写完日志,修改文件习惯的似乎断电,那么通过日志能很快找到需要重新检查的文件。

日志文件系统提高了文件系统对抗非正常卸载的能力,提高了文件系统的稳定性和安全性。
虽然每次执行写操作的时候,都要先写日志。但是日志本身的数据量非常小,所以日志文件系统使用轻微的性能代价保证了文件系统的稳定性,这点代价还是非常值得的。故得到了广泛的应用。
ext3、ext4、reiserfs 、 reiserfs4 、 zfs 、 xfs 、 btrfs、 ntfs 都是日志文件系统。

\section{虚拟文件系统}

磁盘文件系统上保存的文件是实打实的存储在磁盘上的。而虚拟文件系统则不是。虽然也有目录结构,也有文件,但是这些文件的内容可不存在于硬盘上。


\subsection{内核信息窗口/proc和/sys}

procfs是一个虚拟文件系统,标准挂载目录是 /proc ,内核用这个文件系统显示正在运行的进程的信息。每个进程都表现为一个以进程号为名的目录。\footnote{ 这当中有一个特殊的self目录,是一个软连接,它对每个程序来说都是自己。访问/proc/self/相当于访问 /proc/getpid()/。}
在这个目录下,可以查看进程的一些运行时信息。比如占用的内存大小,打开的文件等。

本来 procfs就只是用来透露运行中的进程的信息的。但是在 sysfs 还没有开发出来前,procfs 担当了多重角色。文件系统信息,网络信息,驱动信息都能在这里找到。
就在/proc要演变成大杂烩的时候,这风头被及时制止。Linux新做了一个sysfs,标准挂载目录是/sys。用于向用户程序传递内核信息。udev即使用/sys的信息管理/dev。
/sys下主要提供的信息有总线、驱动、网络、电源、ACPI、文件系统、内核模块等等各类子系统相关信息。
随着时间推移,原先混于/proc的信息都逐步迁移到/sys下,让/proc纯净,只如名字所言,提供运行中的进程的信息。

\subsection{/dev设备文件系统和udev}

UNIX有个传统的“一切都是文件”原则。这个原则的体现就是 /dev 目录。该目录挂载的是一个设备文件系统。在早些时候,是devfs,由内核维护。加载驱动的时候设备文件被创建,卸载驱动的时候设备文件被删除。
设备文件默认的所有者是root:root,权限是0600。不过这种模式有些许弊端:

\begin{itemize}
\item 不便于权限控制。想声卡最佳的所有者应该是audio:audio,权限模式0660。而内核并不被允许访问/etc/passwd文件,这个问题只能通过开机脚本强制执行chown实现了。
\item 即便使用了脚本,热插拔的设备呢?脚本运行的时候设备还没有出现,等用户插入设备的时候脚本已经执行过了。权限模式如何更改?
\item 设备文件由内核创建,驱动需要硬编码设备文件名。如何给设备起别名\footnote{多个设备文件指向同一个设备}?
\item 内核需要驱动才能检测到设备,为了节约内存,用户希望内核不载入没有的设备的驱动。但是不加载驱动就不知道对应的设备是否存在。如何解决\footnote{微软的解决办法是驱动需要“安装”,写死到注册表里,下次开机就知道加载了。}?
\end{itemize}

针对一系列devfs的短处,内核开发者Greg KH\footnote{此人也是Gentoo开发者哦!}开发了udev作为应对方案。

\begin{itemize}
\item /dev 不再挂载devfs,而只是一个临时目录。通常挂载为tmpfs\footnote{参见 \secref{sec:tmpfs}。}。
\item udev 启动后,遍历/sys目录,收集系统信息。依此信息到/dev目录下创建设备文件。并在此过程中分析 udev rules 规则文件,由规则文件控制设备文件的创建过程。udev rules存放在 /lib/udev/rules.d/ /usr/lib/udev/rules.d/ 和 /etc/udev/rules.d/ 下。
\item udev 监听内核事件,出现热插拔事件后可以创建设备文件,并同样使用规则文件套用规则。
\item udev 从sysfs获悉总线信息,然后使用hwids(硬件数据库)\footnote{每一种PCI设备有一个全球唯一的厂商ID和设备ID,此ID在BIOS自检的时候即被收集,所以内核没有驱动也有关于系统上有那些设备的ID号信息(但是不知道具体是什么设备)。用户在自己电脑上执行lspci即可获得此信息哦~。这个厂家ID和设备ID到设备类型和功能对应(包括名字)有一个IEEE维护的数据库。安装 sys-apps/hwids 即可安装此数据库。udev依赖hwids,所以这个包会被自动安装。}即可知道总线上有那些设备并加载对应的驱动。
\end{itemize}

由于udev启动的时候需要解析数千个rules文件,遍历数万个/sys文件,所以启动速度非常慢。大约需要数秒的事件才能完成udev的启动。而/dev如此重要,他必须是init启动的第一个程序。这就大大延迟了开机速度。
所以新捣鼓了一个 devtmpfs 文件系统。结合了devfs和tmpfs的文件系统。这样udev的启动可以被延迟到需要的时候。udev就算不启动,也有devtmpfs提供的设备文件。udev启动后也只需要修改而不是重新创建,所以加快了系统的启动速度。
udev在新版本中已经要求内核必须有devtmpfs支持了。所以编译内核的时候,千万别把devtmpfs选项去掉哦~。

由于udev对开机过程来说是个非常重要的程序,所以现如今已经和systemd合并。udev是systemd的一部分了。

\subsection{虚拟内存盘tmpfs}\label{sec:tmpfs}

\section{网络文件系统}

我们一直在追求网络透明——像本地文件那样操作远程主机上的文件。\footnote{商业上炒概念的人把它叫做“云”。}
X协议让我们的窗口系统网络透明,是不是也有办法能做到文件系统的网络透明呢?

\subsection{NFS}

SUN开发了一个协议RPC\footnote{Remote Procedure Call,远程过程调用,参考 \faqref{FAQ:RPC}}。居然还用这个协议实现了文件系统的read/write/open/close等调用,这就实现了一个网络文件系统(NFS, Network File System)。
NFS是个C/S结构的网络文件系统。提供共享文件的主机是服务器,其他主机作为客户端连接到服务器上使用其共享的文件。访问NFS上的文件于访问本地文件并无二致。在内核,文件系统层向远程主机发起RPC调用而不是转而执行磁盘访问。

使用NFS和其他文件系统一样,都要先``mount"。比如如下的命令将主机``server''\footnote{没有名称的,直接使用IP地址嘛。}的 /usr/portage 挂载到本地的	/usr/portage

\begin{code}
\# mount -t nfs //server/usr/portage	/usr/portage
\end{code}

还可以使用 fstab(5) 做到开机自动挂载。至于服务器端,当然不是随便都能共享的啦!首先要开启nfsd服务,没开服务别人怎么共享。
开了服务,就可以使用 /etc/exports 控制导出(共享)的目录了。

TODO /etc/exports 解释 %TODO 

\subsubsection{Gentoo下安装和使用NFS}
TODO Gentoo 下安装和使用NFSD

\subsection{Windows网上邻居:CIFS}\label{sec:fscifs}

Windows的网上邻居,在资源管理器上使用``\textbackslash\textbackslash对方主机名字\textbackslash共享文件夹名'' 即可访问共享的文件夹。还可以映射到本地驱动器,相当于UNIX下的mount了吧。

网上邻居使用的是CIFS协议,也就是SMB协议。在Linux下实现该协议的软件是samba,samba包含客户端和服务器端,不过内核也实现了一个客户端,使用 mount.cifs 命令挂载。

\begin{code}
\# mount.cifs //server/共享	 /mnt/cifsshare
\end{code}

该命令将名server\footnote{没有名称的,直接使用IP地址嘛。}的主机上共享名称为“共享”的文件夹挂载到本地的 /mnt/cifsshare 上。如果Windows设置了共享密码,mount还会提示输入用户名和密码。
如果不想打命令手工挂载,并且使用KDE或 GNOME这样的桌面环境,在文件浏览器的地址栏\footnote{GNOME默认隐藏掉地址栏,使用Ctrl+L显示。}里输入``\url{smb://server/共享}''即可浏览,这个使用的是samba提供的客户端库实现的。非GNOME或KDE程序无法打开这样的URL,遇到无法打开的情况,还是得内核的支持管用。

\subsubsection{Gentoo下安装和使用Samba}

浏览Windows的共享文件看来非常简单,那么反过来给Windows提供共享呢?诶呀,那简单的,安装samba,把smbd服务加入开机启动列表,配置一下 /etc/samba/smb.conf 就好了。

使用systemd的话,需要使用systemctl enable smbd.service,如果是openrc,则是 rc-update add smbd。

TODO
% TODO 
参考讲解
/etc/samba/smb.conf.default

\section{其他的文件系统}

TODO iso9660 


\subsection{LiveCD的最爱——压缩文件系统squashfs}

TODO mksquashfs 



\subsection{为Flash芯片设计的文件系统}

在嵌入式系统上,通常会使用FLASH芯片作为外部存储设备。FLASH芯片相比硬盘有其固有特征,适用于硬盘的文件系统在FLASH下并不是最合适的。

FLASH的特点:

\begin{itemize}
\item 没有机械运动,随机访问速度很快。
\item 数据访问以块为单位。典型的块大小是4096字节。
\item 数据写入前要先做擦除。整个清空块上的所有数据。即使只修改一个字节也需要整个清除重新写入。
\item 擦写次数有限。FLASH芯片使用的物理原理决定了每块数据区只能进行有限次数的擦除。通常这个次数不会太高,看芯片档次,有一万次到十万次擦写次数不等。
\end{itemize}

因为没有机械运动,所以不用像硬盘那样担心磁盘碎片带来的性能问题。FLASH可以不需要管碎片问题。
写入前要擦除,而且擦除块通常比较大,所以文件系统要设计成“追加”数据而不是在原有的数据上修改。
数据写入次数有限,文件系统必须进行块平衡算法,把数据写入操作平均分配到块中。不能因为一个区块的过度损耗导致整个芯片报废。

专门为FLASH设计的文件系统,要考虑到FLASH的优缺点,扬长避短。
