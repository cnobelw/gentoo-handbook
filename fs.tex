
操作系统内核是一个抽象层,提供统一的接口给应用程序。网络是套接字(socket),对于存储设备,就是文件系统。
UNIX类操作系统使用树形文件系统结构。文件被组织成目录,目录和文件还可以组织成父级目,一直到根目录。

TODO 树形目录插图。 %TODO

文件系统提供了统一的方式组织数据。应用程序和用户不需要关心数据存储在哪个磁盘扇区,只需要使用一个文件名去引用即可。
在内核,同样是提供“文件”这么一个抽象的接口,不同的人提出了不同的数据组织方式。文件系统类型就是文件在磁盘上的具体组织格式。
按照数据存储设备的类型,有块设备文件系统,虚拟文件系统和网络文件系统三大类。硬盘是最主要的块设备,多数块设备文件系统也都是为磁盘设计的。

\section{磁盘文件系统}

组织磁盘扇区存储文件,是内核设计文件系统设计的主要目的。

\subsection{块设备和磁盘}
\subsection{超级块和inode}
\subsection{典型磁盘文件系统举例}
\section{虚拟文件系统}
\subsection{/proc内核信息窗口}
\subsection{/dev设备文件系统和udev}
\subsection{虚拟内存盘tmpfs}
\section{网络文件系统}
\subsection{NFS}
\subsection{Windows 网络邻居 CIFS}
\section{其他的文件系统}
\subsection{LiveCD的最爱——压缩文件系统squashfs}
\subsection{为Flash芯片设计的文件系统}
