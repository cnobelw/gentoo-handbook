

\section{搭建 HTTP 服务器}

首先要做的是先安装一个HTTP服务器。Gentoo在portage里收录了很多HTTP服务器。这里自然说的是LAMP\footnote{Linux、Apache 、 MySQL、 PHP}里的Apache HTTP Server。

\subsection{安装Apache}

\begin{code}
\# emerge apache -av
\end{code}

由于apache高度模块化,各个功能被拆分到了不同的模块里。而Gentoo独有的USE机制就允许用户只编译需要的模块。
安装 apache 会出现类似下面的提示
\begin{code}
[ebuild  N     ] app-admin/apache-tools-2.4.3-r1  USE="ssl" 4,453 kB

[ebuild  N     ] www-servers/apache-2.4.3:2  USE="ssl -debug -doc -ldap (-selinux) -static -suexec -threads" APACHE2\_MODULES="actions alias auth\_basic authn\_alias authn\_anon authn\_core authn\_dbm authn\_file authz\_core authz\_dbm authz\_groupfile authz\_host authz\_owner authz\_user autoindex cache cgi cgid dav dav\_fs dav\_lock deflate dir env expires ext\_filter file\_cache filter headers include info log\_config logio mime mime\_magic negotiation rewrite setenvif socache\_shmcb speling status unique\_id unixd userdir usertrack vhost\_alias -access\_compat -asis -auth\_digest -authn\_dbd -cache\_disk -cern\_meta -charset\_lite -dbd -dumpio -ident -imagemap -lbmethod\_bybusyness -lbmethod\_byrequests -lbmethod\_bytraffic -lbmethod\_heartbeat -log\_forensic -proxy -proxy\_ajp -proxy\_balancer -proxy\_connect -proxy\_ftp -proxy\_http -proxy\_scgi -reqtimeout -slotmem\_shm -substitute -version" APACHE2\_MPMS="-event -itk -peruser -prefork -worker" 24 kB
\end{code}

可以在make.conf里设定APACHE2\_MPMS和APACHE2\_MODULES的值。初次配置apache的同学可能要犯傻了,这么多模块! 我咋知道我需要哪个呢?

拿不准的情况下就全编译了吧(笑)。

\subsection{添加开机自启动Apache}

如果你用的是OpenRC,使用下面的命令添加到开机自启动

\begin{code}
\# rc-update add apache2 default\\
\# /etc/init.d/apache2 start
\end{code}

如果用的是 SystemD, 则简单如下\footnote{记得添加systemd overlay, 并安装 systemd-units的时候开启server这个USE flag。}

\begin{code}
\# systemctl enable apache2.servie\\
\# systemctl start apache2.service
\end{code}

\textbf{如果}将来的某一天 Apache 获得了 Socket Activation 支持,则可以用

\begin{code}
\# systemctl enable apache2.socket\\
\# systemctl start apache2.socket
\end{code}

这个时候Apache将会延迟到第一个客户链接的时候才启动。当然也可以在第一个客户连接上之前启动。

\begin{code}
\# systemctl enable apache2.socket\\
\# systemctl start apache2.socket\\
\# systemctl enable apache2.service\\
\# systemctl start apache2.service
\end{code}

\subsection{php支持}

要使用PHP,首先安装php,同时别忘记启用apache2 这个USE标记。

\begin{code}
\# USE=apache2 emerge php
\end{code}

或者将 dev-lang/php apache2 写入 /etc/portage/package.use。
然后编辑/etc/conf.d/apache2,在APACHE2\_OPTS= 后面追加 \texttt{"-D PHP5"}

最后重启 apache:

\begin{code}
\# systemctl restart apache2

或者

\# /etc/init.d/apache2 restart

\end{code}

\begin{notice}
php 的USE标记也是相当的多,而且还有相互依赖关系。开启一个USE而不开启另一个USE或将导致portage报错。自求多福吧。(笑)
依据提示慢慢的修改。
\end{notice}

\section{数据库}

MySQL 是最流行的开源数据库。虽然被Oracle收购后前景似乎不那么没明朗了。不过仍然是最流行的数据库。

少废话,先安装:

\begin{code}
(查看可用的 USE flags) \\
\# emerge --pretend --verbose mysql \\
(安装 MySQL) \\
\# emerge mysql
\end{code}

记住开启所有自己先要开启的功能。

如果这是你第一次安装 mysql , portage会在接近完成的最后输出这样的话:

\begin{code}
You might want to run: \\
"emerge --config =dev-db/mysql-[version]" \\
if this is a new install.
\end{code}

也就是利用 “\# emerge --config  dev-db/mysql” 这条命令进行配置。设定root密码,建立初始数据库。

\begin{code}
\# emerge --config  dev-db/mysql
\end{code}

按照提示设定密码即可。

接着就和 apache 一样了,分为 OpenRC 模式和 systemd 模式,分别有两种不同的办法设定为开机启动。

\begin{code}
\# rc-update add mysql default\\
\# /etc/init.d/mysql start
\end{code}

或者 systemd 模式

\begin{code}
\# systemctl enable mysql.service\\
\# systemctl start mysql.service
\end{code}

或者如果哪天mysql的systemd socket activation模式并入主分支了,可以这么干:

\begin{code}
\# systemctl enable mysql.socket\\
\# systemctl start mysql.socket
\end{code}

好处就是,不需要为Web服务器和MySQL安排谁先谁后的启动次序了。第一个到mysql的连接建立的时候mysql随即被启动。
如果担心第一次使用的等待时间过长,可以使用socket activation的同时启用mysql.service.


\begin{code}
\# systemctl enable mysql.socket\\
\# systemctl start mysql.socket

\# systemctl enable mysql.service\\
\# systemctl start mysql.service
\end{code}

这样即获得了不需要操心Web服务器和MySQL谁先谁后的问题,同时也没有第一次使用的延时问题了。

\section{文件共享——FTP和Samba}

我们可以使用FTP来传输文件,也可以使用网上邻居 —— Samba。

\subsection{FTP}

FTP 是在internet上进行文件传输的协议。从名字上来说FTP就是File(文件)Transfer(传输)Protocol(协议)的缩写。FTP的实现非常多,不过似乎大家都用一个叫 Very Secure FTP Daemon 的软件。

第一步先安装它。
\begin{code}
\# emerge -av vsftpd
\end{code}

配置允许匿名访问,匿名用户的根目录是 /home/ftp:

编辑配置文件/etc/vsftpd/vsftpd.conf
\begin{code}

listen=YES

local\_enable=NO

anonymous\_enable=YES

write\_enable=NO

anon\_root=/home/ftp
\end{code}

接着是配置开机启动。

这是openrc下的配置办法:
\begin{code}
\# rc-update add vsftpd default

\# /etc/init.d/vsftpd start
\end{code}

systemd 下则这样配置:
\begin{code}
\# systemctl enable vsftpd

\# systemctl start vsftpd
\end{code}

\subsection{Samba}

如果仅仅为了和同事或家庭成员共享一下文件,大可不必搭建一个FTP服务器。使用“网上邻居”即可。
在Linux下借助Samba这个网上邻居的开源实现就可以和Windows主机相互访问了。

恩,当然首先还是先安装:
\begin{code}
\# emerge samba
\end{code}

samba 分客户端和服务器端两个。客户端除了samba本身提供的一个命令行界面的 smbclient外,还有内核实现的 CIFS\footnote{参加 \secref{sec:fscifs}}。 
各个桌面环境还有利用Samba提供的libsamba库实现了客户端,直接在文件管理器的地址栏里输入网上邻居的地址\footnote{$\backslash\backslash$主机名$\backslash$共享名 }即可访问。

KDE 环境下,要实现在文件管理器地址栏里直接输入网上邻居地址,要为 kdebase-kioslaves 加入 samba USE flag。 而要编译出libsamba就需要为samba加入client这个USE flag。
然后重新编译 kde-base/kdebase-kioslaves —— 如果开始并没有编译进 samba 支持的话。
\begin{code}
\# echo net-fs/samba client >> /etc/portage/package.use/net-fs

\# echo kde-base/kdebase-kioslaves samba >> /etc/portage/package.use/kde-base

\# emerge kde-base/kdebase-kioslaves --oneshot

\end{code}

GNOME使用gvfs,只要gvfs支持了samba即可让所有gnome程序支持samba,当然也包括gnome的御用文件浏览器nautilus了。

\begin{code}
\# echo net-fs/samba client >> /etc/portage/package.use/net-fs

\# echo gnome-base/gvfs samba >> /etc/portage/package.use/gnome

\# emerge gnome-base/gvfs -av --oneshot
\end{code}

这样nautilus就支持在地址栏直接输入网上邻居的地址访问了。

那么剩下的问题,如何配置让Windows访问Linux呢?

samba 的配置文件为 /etc/samba/smb.conf

smb.conf 的格式为ini格式,也就是

\begin{code}
[共享名]

path = 本机路径\\
read only = yes或no\\
guest ok = yes或者no\\
printable = yes或者no
\end{code}

这样的格式。
guest ok 控制是否允许匿名访问,read only控制是不是只读,printable控制是不是可打印。

其中用三个特素的节[global], [homes] 和 [printers]。 他们的配置选项比较特素化。

global用于全局配置。global里配置会编程其他节的默认配置。例如在global下配置 read only=no,则所有共享默认可写除非设置了read only = yes。

homes 则是账户目录模板。比如用户请求的共享名为 jhonsmith, 但是 smb.conf 并没有包含 [jhonsmith] , samba 并会查找本地账户是否有jhonsmith,恰好本机有一个 jhonsmith 账户,那么 [homes] 内的设置将会作为模板用来建立  jhonsmith  这个共享目录。

[printers] 用户设置打印机,使用printable = yes

\section{共享打印机}

办公室/家里有很多电脑?只有一台打印机?那么需要共享打印机咯!

\subsection{使用CUPS共享打印机}

要共享打印机,需要一台电脑作为打印服务器。既然CUPS是一种打印服务,它能接受本机程序的请求,也一定能用接受网络上的请求。要接受网络请求,第一个想到的问题就是\textbf{权限}。如果没有权限控制,你的打印机岂不是要白白的被人使用?白白浪费许多墨水。

CUPS默认禁止网络访问,所以你所需要做的就是配置好它。

现在还在使用文本编辑器?弱爆了。咱有 kde-base/print-manager ! 先安装了它\footnote{它可能已经作为KDE的一部分安装了,吼吼}。

\begin{code}
\# emerge kde-base/print-manager
\end{code}

打开KDE的设置,选择打印机。输入root密码完成授权。

\chatu{kdesystemsettingsprint}{KDE设置-打印机}

然后点击“系统首选项”->“共享链接到此系统的打印机”。如图 \ref{fig:kdeprintmanagerallowshare} 所示。

\chatu{kdeprintmanagerallowshare}{KDE打印机设置-允许共享打印机}

然后将需要共享的打印机的“共享此打印机”的复选框选中即可。

接下来是客户端的设置。

\begin{enumerate}
\item{}浏览器打开\url{http://localhost:631/admin}

\item    点击添加打印机“Add Printer”
\item    选择 Internet Printing Protocol (http) 
\item   点继续“continue”
\item   输入打印机的地址,比如 http://printerserver:631/printers/Canon\_MP288
\item   点继续“continue”
\item   选择驱动,然后完成。
\end{enumerate}

\begin{notice}
像 http://printerserver:631/printers/Canon\_MP288 这样的 URL 如何获得?

可以打开 http://192.168.1.1:631/printers/ 浏览该主机共享的打印机。
\end{notice}

如果是Windows客户端,同样选择网络上的打印机,然后使用该URL即可。如图 \ref{fig:winaddprinter} 所示。

\chatu[scale=0.6]{winaddprinter}{Windows添加网络打印机}

\subsection{使用Samba共享打印机}

Samba不仅仅可以共享文件,还可以为Windows主机提供打印机共享。因此除了让Windows直接使用CUPS外,还可以(对Windows来说)以更友好的网上邻居提供打印机共享。

\begin{code}
[global]

		map to guest = bad user <== 这是为了能不用用户名密码就自动认证到Guest账户

		.....其他省略.....
		
		\#这里列出的则是打印机相关的资料啊!
		load printers = yes
		
		printing = cups

[printers] <==这个分享的资源名称一定是printers

		comment = All Printers
		
		path = /var/spool/samba
		
		printer admin = root
		
		create mask = 0600
		
		guest ok = Yes
		
		printable = Yes
		
		use client driver = Yes
		
		public = yes
		
		browsable = yes
		
		guest ok = yes
		
		.....其他省略..... 
         
\end{code}

如注释里说的,打印机资源的名称必须是printers。


然后Windows用户即可正常的在资源管理器里里浏览网上邻居,就能直接在网上邻居里添加打印机了。如图 \ref{fig:winbroweprinter}。

\chatu[scale=0.55]{winbroweprinter}{浏览网上邻居的打印机}

需要安装Windows打印机驱动哦。
