

\begin{landscape}
{
\renewcommand{\clearpage}{}
\renewcommand{\newpage}{}
\chapter{名称缩写}
}

{
因为在正文中多数引用了英文缩写,虽然都在脚注或者括号的形式注明了非缩写形式,但是还是觉得有必要在这里重写列举一下正文里用到的引用。表格比较宽,所以使用横式排版,请谅解。
}
\begin{center}


\tablefirsthead{
\multicolumn{1}{c|}{缩写} & \multicolumn{1}{c|}{英文原词} & \multicolumn{1}{c}{含义} \\
\hline
}

\tablehead{
\multicolumn{1}{c|}{缩写} & \multicolumn{1}{c|}{英文原词} & \multicolumn{1}{c}{含义} \\
\hline
}

\tabletail{
\hline
\multicolumn{1}{c|}{缩写} & \multicolumn{1}{c|}{英文原词} & \multicolumn{1}{c}{含义} \\
}


\tablelasttail{
\hline
\multicolumn{1}{c|}{缩写} & \multicolumn{1}{c|}{英文原词} & \multicolumn{1}{c}{含义} \\
}

\begin{supertabular}{l|c|r}
API & Application Programming Interface & 应用程序编程接口 \\
AT\&T & American Telephone and Telegraph Company & 美国电话电报公司\\
BIOS & Basic Input Output System & 级别输入输出系统\\
BSD & Berkeley Software Distribution & 伯克利软件套件 \\
CSMA/CD & Carrier Sense Multiple Access/Collision Detect & 带冲突检测的载波监听多路访问  \\
CTSS & Compatible Time Sharing System & 相容分时系统 \\
Distro & Distribution & 英文简写的发行版 \\
EFI & Extensible Firmware Interface & 可扩展固件接口 \\
EPON & Ethernet Passive Optical Network & 以太无源光网络\\
FTP & File Transter Protocol & 文件传输协议 \\
GCC & GNU Compiler Collection & GNU 编译器套件 \\
GNU & GNU is Not Unix & GNU不是UNIX \\
GPL & (GNU) General Public License & (GNU)通用公共许可协议 \\
GPT & GUID Partition Table & 唯一标识符磁盘分区表,EFI/UEFI所使用\\
GRUB & (GNU) GNU GRand Unified Bootloader & (GNU) 统一引导管理器 \\
HTTP & Hyper Text Transfer Protocol & 超文本传输协议 \\
IBM & International Business Machine & 国际商用机器公司 \\
IEEE & Institute of Electrical and Electronics Engineers  & 电气电子工程师学会\\
IO & Input Output & 输入输出 \\
IPC & Inter Process Communication & 进程间通信 \\
KISS & Keep It Simple, Stupid& 保持简单 ,傻瓜 \\

LFS & Linux From Scratch & 从头开始构建Linux \\
MBR & Master Boot Record & 主引导记录。硬盘的第一个扇区 \\
MMU & Memory Management Unit & 内存管理单元 \\

MULTICS & MULTiplexed Information and Computing System & 多路信息计算系统 \\
POST & Power On Self Test & 加电自检\\
POP3 & Post Office Protocol version 3 & 邮局协议3\\

POSIX & Portable Operating System Interface (of uniX) & 可移植操作系统接口 \\
PPP & Point-to-Point Protocol & 点对点协议 \\
PPPoE & PPP over Ethernet& 以太网上的点对点协议 \\
RAM & Radom Access Memory & 随机访问存储器\\
RHEL & RedHat Enterprise Linux & 红帽企业版Linux\\
ROM & Read Only Memory & 只读存储器\\
SMTP & Simple Mail Transfer Protocol & 简单邮件传输协议 \\
SSL & Secure  Socket Layer & 安全套机字层 \\
%Sun & Stanford University Network & 斯坦福大学网络部・太阳微公司 \\

SysV & System V UNIX & 第五版 UNIX , 贝尔实验室发行 \\
UEFI & Unified Extensible Firmware Interface & 统一可扩展固件接口 \\
UNICS & UNiplexed Information and Computing System & 单路信息计算系统\\
\end{supertabular}
\end{center}
\end{landscape}