
\section{软件的分发}

除了写给自己用的软件,大部分软件都需要向最终用户进行分发。软件的分发不仅仅是一次下载,一份拷贝。如何发行软件也是很考究的活。
安装发布内容的形式可以分为二进制发布和源码发布。

闭源软件没得说,只有二进制发布。开源软件通常使用源码发布,并为流行的发行版提供预编译的二进制发布。

商业软件使用二进制发布,通常还会带上一个安装程序。用户只要执行这个安装程序就可以自动的将软件安装到指定位置,调节好系统配置,建立快捷方式等等操作。
安装程序的傻瓜方式就是 “下一步”。使用Windows的人,不知不觉的养成了一路Next的习惯。有恶意程序就专门针对一路Next的人的习惯偷偷的给安装上一些广告插件。

Linux下的软件通常以源码形式发布。源码常常以tar打包并用gz或者bzip2压缩,扩展名 .tar.gz 或者 .tar.bz2。发布的文件名通常为 “软件名-版本.tar.bz2”。
用户将源码下载后,就进入了编译阶段。软件会在压缩包内带一份文档告诉用户如何编译,或者在官方网站上提供编译指导。

\section{软件的编译}
源码发布,不仅仅是把源代码打包发布完事,还需要带上一套构建脚本。但凡一个真正工作的现实软件,都不会只有一个 .c 文件完事。有的大型软件包含上万个文件。
如此众多的文件,一个一个手工调用gcc命令去编译显然不人性化,也非常烦琐。所以但凡一个软件项目,都有一套编译系统。

编译系统不仅仅用于从源码从头构建可执行文件,还包括追踪文件的修改,再次编译的时候只重新编译改动的部分。这样可以极大的加快开发的进程。因为开发者做一个小修改不需要重新编译所有的文件。
基于这样的目的,UNIX界发明了Makefile。

\subsection{Makefile}

Makefile是一个基于时间戳的依赖追踪器。Makefile 定义多个目标和依赖。并在目标的时间戳比依赖旧的情况下调用定义的命令行更新目标。
比如下面的一个Makefile文件:

\begin{code}
helloworld: helloworld.c\\
\qquad gcc helloworld.c -o helloworld
\end{code}

意思是当helloworld的最后修改时间比helloworld.c旧,或者helloworld文件不存在的时候,调用`gcc helloworld.c -o helloworld'命令更新helloworld文件。

\begin{notice}
但是直接执行 make , 则不会更新 helloworld, 必须执行 make helloworld。 因为 helloworld 不是默认更新目标。直接执行 make 只会更新默认的目标。默认更新目标是 all, 所以需要这样写

\begin{code}
all: helloworld
\\
\qquad
\\

helloworld: helloworld.c\\
\qquad gcc helloworld.c -o helloworld
\end{code}

all依赖helloworld,所以更新 all 前要更新 helloworld, 这样直接执行 make 即可更新helloworld了。

\end{notice}

通常发布的Makefile还定义一个特殊的目标,  install: 直接执行 make install 就可以把编译好的可执行文件拷贝到系统的 bin 目录下。

一旦项目变大,手写的Makefile一起膨胀。逐个的追踪文件的变化,并手工更新相互间的依赖关系变得非常的头疼。所以,最好的办法是有个工具能自动的生成Makefile。
这样很多软件都可以有统一的make行为了,比如支持 make test,  make install, make uninstall, make dist-bzip2 自动打包为源码发布包,等等。

利用更简单的规则生成符合标准的Makefile的工具有很多,最常用的莫过于autotools了,这个是软件发布的事实标准。除了autotools,还有scons, cmake, qmake 等。
这里着重介绍常用的autotools和cmake的使用。

因为本书并不是开发手册,所以并不打算教大家如何编写Makefile, 如何编写autotools的规则,如何编写CMakeLists.txt。需要知道这些详细规则的同学可以参考各自工具的手册。

\subsection{autotools}

autotools 并不是一个工具,而是一组工具的集合。autotools 在开发阶段需要 m4, automake , autoconf 三个工具的配合,最终生成的是一个名为 configre 的 shell 脚本和一系列的Makefile.in文件。生成的configure 脚本和Makefile.in被打包到最终发布的源码中。因此\textbf{用户并不需要安装autotools工具集\footnote{这是autotools非常重要的优点,最终发布的源码包只依赖shell脚本和make工具。并不依赖autotools本身。}}。用户需要执行configure脚本从Makefile.in生成Makefile。然后使用make编译即可。

典型的使用 autotools 编译工具的源码包的编译流程为:

\begin{code}
\#tar -xvf somepkg-ver.tar.bz2

\#cd somepkg-ver

\#./configure --prefix=/usr % [--prefix=/installpath --enable-XXX --disable-XXX --with-XXX --without-XXX] 

\#make

\#sudo make install
\end{code}

configure、make、make install 又被称为软件编译三部曲。括弧笑。

代码的编译期配置都发生在调用 ./configure 脚本的时候。所有的 ./configure 脚本都支持 --prefix 指定安装路径。通常将程序安装到 /usr 下。
autotools 的规则是这样的

\begin{enumerate}
\item 可执行文件安装到 \$\{prefix\}/bin 下
\item 库文件安装到 \$\{prefix\}/lib \footnote{如果指定了 --libdir=XXX , 则是安装到 \$\{libdir\} 下 } 下
\item 头文件安装到 \$\{prefix\}/include 下
\item 数据文件安装到 \$\{prefix\}/share 下
\end{enumerate}

所以指定了 --prefix 就把其他相对路径都确定下来了。如果不指定 prefix , 默认则是 /usr/local。

autotools 天生支持交叉编译。要交叉编译,只需要使用 --host=XXX 参数即可。
GNU为一个平台定义了一个叫 host triplets 的东西,一个平台使用 cpuarch-platform-kernel 这样的三组来唯一确定。比如64位PC上跑的linux系统就是 x86\_64-pc-linux-gnu。 x86\_64 是 CPU 架构, pc-linux 指的是操作系统内核\footnote{一些发行版使用 unknow-linux},gnu指的是libc的牌子,表示使用glibc\footnote{嵌入式系统通常是uclibc}。这样处理这样一个平台的编译工具都使用 x86\_64-pc-linux-gnu- 前缀,比如 x86\_64-pc-linux-gnu-gcc , x86\_64-pc-linux-gnu-g++,  x86\_64-pc-linux-gnu-ld , x86\_64-pc-linux-gnu-ar . 如果指定了 --host=x86\_64-pc-linux-gnu-,  则 configure 生成的 Makefile 了默认就把各种编译工具设置为 x86\_64-pc-linux-gnu- 前缀的版本。


\subsection{cmake}

autotools 对最终用户非常友好,可是开发者需要做大量的工作。显然搞明白 m4 automake autoconf (甚至还有 libtool) 相互之间的复杂关系非常令人头痛。所以 KDE 果断迁移到了 cmake 构建工具。连 KDevelop 这种重量级的IDE都放弃了原生autotools支持,转而支持cmake。

cmake 使用一个 CMakeLists.txt 文件来记录项目的编译配置。开发者通过编写 CMakeLists.txt 及其简单的规则, 然后由 cmake 处理成 Makefile。 cmake 除了能将 CMakeLists.txt 转化为 Makefile, 还可以转化为 Visual Studio 的工程文件,生成 Eclipse 的工程文件,CodeBlocks的项目文件。是个对 IDE 比较友好的 工具。



\section{发行版的包管理}\label{section:pkgmgr}
\section{portage和emerge}\label{sec:emerge}

\section{什么是overlay,如何使用}\label{sec:overlay}

/usr/portage下存放了数万个ebuild文件,这些文件按照预定的格式组织,形成了Gentoo的软件仓库,又称portage树。
使用命令{\tt emerge --sync}可以方便的将本地的portage树与Gentoo官方发布的portage树进行增量同步。

portage里虽然有数万个\footnote{截至到本书写作的时候有32225个ebuild共计15731个软件。}ebuild,但是总有软件是游离在portage之外的。
总要有机制可以添加第三方仓库吧?这样有的特殊软件可以放到第三方仓库中,想要使用的人添加第三方仓库即可使用emerge安装这些软件。
用户还可以自己维护一个给自己用的仓库,编写自己用的ebuild,这样就有可以更强大的定制能力了。
Gentoo使用overlay来支持这种需求。

所谓overlay就是结构和portage一样的一个目录,该目录里存放的ebuild也会被emerge使用。如果overlay里的版本和portage里的一样高,则overlay里的ebuild有更高的优先级。
overlay通常包含portage没有的软件,或者相同版本但是打上不同的补丁以提供增强功能。只要在 make.conf 中设定 PORTDIR\_OVERLAY 为一个overlay的路径即可使用该overlay。
overlay可以有多个,亦即PORTDIR\_OVERLAY可以设定多个路径,以空白隔开(通常是每行写一个overlay,参考 /var/lib/layman/make.conf),相互之间的优先级看它在PORTDIR\_OVERLAY变量中出现的次序。后出现的优先级高。

\subsection{用layman管理overlay}

现成的overlay被gentoo官方收录到了一个xml文件中:

\url{http://www.gentoo.org/proj/en/overlays/repositories.xml}

使用layman(8)工具即可自动获取这些overlay。这些overlay使用版本控制工具\footnote{参考 \chapref{chap:VCS}},用户也使用版本控制工具获得这些overlay。
所以layman依赖一些版本控制工具,可以用USE控制layman依赖的版本控制工具。

\begin{code}
\#USE="git subversion" emerge layman
\end{code}

这样会将 git 和 subversion (大部分overlay使用git和svn)作为 layman 的依赖而被自动安装(如果还没安装的话)。没有安装的版本控制工具将无法添加使用该工具管理的overlay。

在 make.conf\footnote{ /etc/portage/make.conf 别忘记哦!} 里最后加入一行 “{\tt source /var/lib/layman/make.conf}”(不包括引号), 然后就可以使用 layman -a overlay名字 添加指定名字的overlay了。

\begin{insertnote}
注意:/var/lib/layman/make.conf只有在添加了第一个overlay后才会生成。所以可以在添加一个overlay后再加入对该文件的source操作。
\end{insertnote}

\begin{example}{添加 gentoo-zh overlay。}
\begin{code}
\#layman -f \\
\#layman -a gentoo-zh
\end{code}

第一次使用layman需要layman -f,这一步到gentoo.org下载repositories.xml文件。如果已经下载过可以跳过这条命令。

\end{example}

\subsubsection*{列出所有的overlay}

\begin{code}
\#layman -L
\end{code}

\subsubsection*{列出已经添加的overlay}

\begin{code}
\#layman -l
\end{code}

\subsubsection*{添加一个overlay}

\begin{code}
\#layman -a <overlay名字>
\end{code}

\subsubsection*{删除一个overlay}

\begin{code}
\#layman -d <overlay名字>
\end{code}

\subsubsection*{更新一个overlay}

\begin{code}
\#layman -s <overlay名字>
\end{code}

\subsubsection*{一次性升级所有overlay}

\begin{code}
\#layman -S
\end{code}
