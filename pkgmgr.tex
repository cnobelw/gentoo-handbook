
\chapter{软件管理}\label{软件管理}


\section{软件的分发}
\section{软件的编译}
\section{发行版的包管理}\label{section:pkgmgr}
\section{portage和emerge}\label{sec:emerge}

\section{什么是overlay,如何使用}\label{sec:overlay}

/usr/portage下存放了数万个ebuild文件,这些文件按照预定的格式组织,形成了Gentoo的软件仓库,又称portage树。
使用命令{\tt emerge --sync}可以方便的将本地的portage树与Gentoo官方发布的portage树进行增量同步。

portage里虽然有数万个\footnote{截至到本书写作的时候有32225个ebuild共计15731个软件。}ebuild,但是总有软件是游离在portage之外的。
总要有机制可以添加第三方仓库吧?这样有的特殊软件可以放到第三方仓库中,想要使用的人添加第三方仓库即可使用emerge安装这些软件。
用户还可以自己维护一个给自己用的仓库,编写自己用的ebuild,这样就有可以更强大的定制能力了。
Gentoo使用overlay来支持这种需求。

所谓overlay就是结构和portage一样的一个目录,该目录里存放的ebuild也会被emerge使用。如果overlay里的版本和portage里的一样高,则overlay里的ebuild有更高的优先级。
overlay通常包含portage没有的软件,或者相同版本但是打上不同的补丁以提供增强功能。只要在 make.conf 中设定 PORTDIR\_OVERLAY 为一个overlay的路径即可使用该overlay。
overlay可以有多个,亦即PORTDIR\_OVERLAY可以设定多个路径,以空白隔开(通常是每行写一个overlay,参考 /var/lib/layman/make.conf),相互之间的优先级看它在PORTDIR\_OVERLAY变量中出现的次序。后出现的优先级高。

\subsection{用layman管理overlay}

现成的overlay被gentoo官方收录到了一个xml文件中:

\url{http://www.gentoo.org/proj/en/overlays/repositories.xml}

使用layman(8)工具即可自动获取这些overlay。这些overlay使用版本控制工具\footnote{参考 \chapref{chap:VCS}},用户也使用版本控制工具获得这些overlay。
所以layman依赖一些版本控制工具,可以用USE控制layman依赖的版本控制工具。

\begin{code}
\#USE="git subversion" emerge layman
\end{code}

这样会将 git 和 subversion (大部分overlay使用git和svn)作为 layman 的依赖而被自动安装(如果还没安装的话)。没有安装的版本控制工具将无法添加使用该工具管理的overlay。

在 make.conf\footnote{ /etc/portage/make.conf 别忘记哦!} 里最后加入一行 “{\tt source /var/lib/layman/make.conf}”(不包括引号), 然后就可以使用 layman -a overlay名字 添加指定名字的overlay了。

\begin{insertnote}
注意:/var/lib/layman/make.conf只有在添加了第一个overlay后才会生成。所以可以在添加一个overlay后再加入对该文件的source操作。
\end{insertnote}

\begin{example}{添加 gentoo-zh overlay。}
\begin{code}
\#layman -f \\
\#layman -a gentoo-zh
\end{code}

第一次使用layman需要layman -f,这一步到gentoo.org下载repositories.xml文件。如果已经下载过可以跳过这条命令。

\end{example}

\subsubsection*{列出所有的overlay}

\begin{code}
\#layman -L
\end{code}

\subsubsection*{列出已经添加的overlay}

\begin{code}
\#layman -l
\end{code}

\subsubsection*{添加一个overlay}

\begin{code}
\#layman -a <overlay名字>
\end{code}

\subsubsection*{删除一个overlay}

\begin{code}
\#layman -d <overlay名字>
\end{code}

\subsubsection*{更新一个overlay}

\begin{code}
\#layman -s <overlay名字>
\end{code}

\subsubsection*{一次性升级所有overlay}

\begin{code}
\#layman -S
\end{code}
