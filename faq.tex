\renewcommand{\thesection}{\arabic{section}}

\chapter{FAQ}\label{FAQ}

\section{什么是 LiveCD ?}\label{FAQ:LiveCD}
一种让 Linux 系统脱离硬盘,只需要一张 DVD 或者 CD 运行的技术。使用者只需要插入一张 LiveCD 就可以使用 Linux  系统了。当然,由于 CD-ROM 是只读的,所以对 LiveCD 系统本身所做的更改重启后都会消失。由于 Linux 发行版的膨胀,一张 CD-ROM 无法装下,于是就有了 LiveDVD , 除了容量增加外,和LiveCD 是一致的。此外还有 LiveUSB , 也就是使用 USB 存储设备的 liveCD。用法和 LiveCD 并无二致。

\section{如何使用 LiveCD。}\label{FAQ:UseLiveCD}

\section{UEFI和BIOS是什么?}


\section{什么是API?。}\label{FAQ:API}

\section{Shell 是什么?}
操作系统是由内核和一系列的系统程序组成的。

\section{什么是进程和线程?}\label{FAQ:Process}

\section{什么是内存分页?}\label{FAQ:Paging}