\renewcommand{\thesection}{\arabic{section}}

\chapter{FAQ}\label{chap:FAQ}

\section{什么是 LiveCD ?}\label{FAQ:LiveCD}
一种让 Linux 系统脱离硬盘,只需要一张 DVD 或者 CD 运行的技术。使用者只需要插入一张 LiveCD 就可以使用 Linux  系统了。当然,由于 CD-ROM 是只读的,所以对 LiveCD 系统本身所做的更改重启后都会消失。由于 Linux 发行版的膨胀,一张 CD-ROM 无法装下,于是就有了 LiveDVD , 除了容量增加外,和LiveCD 是一致的。此外还有 LiveUSB , 也就是使用 USB 存储设备的 liveCD。用法和 LiveCD 并无二致。

\section{如何使用 LiveCD。}\label{FAQ:UseLiveCD}

\section{UEFI和BIOS是什么?}


\section{什么是API?。}\label{FAQ:API}

\section{Shell 是什么?}
操作系统是由内核和一系列的系统程序组成的。

\section{什么是进程和线程?}\label{FAQ:Process}

\section{什么是内存分页?}\label{FAQ:Paging}

\section{如何不重新登录切换到root帐号?}\label{FAQ:su}

正如 \secref{sec:usersandgroups} 所讲,UNIX支持多账户登录。如果当前使用普通帐号登录,而某些命令需要使用root帐号的时候,就需要用到su这个命令了。
使用su命令的前提是加入wheel组,否则不能使用su切换到root。 “su”切换到root帐号,但是没有切换环境变量和工作目录,亦没有执行/root/.bash\_profile。{}“\hbox{su -}” 能切换到 /root 目录,提供类似直接登录的环境。用户很可能需要的就是带 - 参数的su。

su 还可以切换到其他用户,如“su smith - ”可以切换到smith帐号上。

另外一个临时使用root权限的办法就是在执行的命令前加 sudo 。 sudo 需要用户加入wheel组并在 /etc/sudoers 加入 “\%wheel ALL=(ALL) ALL” 一行。
sudo 通常来说只应该授权给信任的普通帐号 —— 比如管理员自己给自己设定的日常登录帐号(如果是桌面用户,自然就是自己用来登录桌面的帐号) —— 并且为普通用户设定好强壮的密码。

\section{内核参数}\label{FAQ:kernelparamter}


\section{Initramfs是什么} \label{FAQ:initramfs}

Linux启动过程中内核主动做的最后一件事情就是执行/sbin/init,那么倒数第二件事情必然是挂载根目录。
挂载根目录有2个前提条件:找到根分区和支持根分区所使用的文件系统。
找到根分区,第一点,需要其所在设备的驱动。第二点就是内核需要知道哪个是根分区。这个可以通过内核参数 root= 设定。

Linux下驱动可以编译进内核,也可以编译为模块动态加载。如果恰巧磁盘设备的驱动被编译成了模块,就出现“鸡生蛋和蛋生鸡”的问题了:不访问磁盘就无法加载模块,而不加载模块有没有磁盘的驱动。这对于自己编译内核的童鞋不成问题。只要知道自己的SATA/IDE控制器的型号,将对应的驱动编译进内核就解决了。但是如果是打算制作“通用内核”的发行版呢?

文件系统也是同样的问题,没有文件系统驱动就无法挂载分区访问上面的文件,无法访问文件就无法加载模块。这对于自己编译内核的童鞋不成问题。只要将自己根分区所使用的文件系统编译进内核就解决了。但是如果是打算制作“通用内核”的发行版呢?将所有的文件系统都编译进内核么?

initramfs就是为解决这个“鸡生蛋和蛋生鸡”的问题诞生的。

GRUB加载内核的时候通过initrd 命令加载一个initramfs,该initramfs就会成为内核的根分区\footnote{事实上是Linux内核拥有一个不挂载就存在的rootfs文件系统,Linux启动过程中将initramfs的内容解压到rootfs中。作为理解,可以认为initramfs挂载成Linux的临时根目录。}。内核初始化完毕后直接执行initramfs中的/sbin/init而不理会root=命令行参数。由initramfs中的/sbin/init程序负责加载磁盘驱动,查找根分区,挂载根分区到 /sysroot 并调用pivot\_root()\footnote{使用命令 man 2 pivot\_root查看介绍。}将找到的根目录替换为真正的根目录。然后执行真正的/sbin/init程序。

initramfs是一个目录的压缩包。由dracut程序生成。initramfs用于寻找真正的根目录时的临时根目录,通常只包含一些必要的驱动和一些用于查找根分区的脚本,但有时候可以成为一个完成的系统的根目录,这样的系统可以脱离磁盘运行。比如Fedora的安装DVD,使用一个巨大initramfs作为安装环境的根目录。该initramfs包含的是一个真正的OS,带有完整的工具。


\section{汉化man手册}\label{FAQ:zhman}

man使用程序使用nroff程序格式化man手册,然后使用less向屏幕分页输出。为什么呢?因为man手册的显示会随着终端的宽度不同而变化。所以man手册并不是使用原始的ascii文本文件编写的。
man手册源文件使用非常隐晦难懂的一些标记,这些标记借助nroff程序格式化后就获得了适合特定宽度的终端显示的文本。正因为如此,所以man手册能在各种大小的终端上显示而不破坏风格。

GNU使用的nroff是groff(GNU出品,带个g是常有的事情),但是groff并不支持UTF-8编码的文本。所以导致man手册中不能出现非英文字符。所以有人对groff进行了修改,添加了UTF-8支持,这就是groff-utf8。要想显示中文man手册,按照man-pages-zh\_CN这个中文手册之外,还得安装groff-utf8并修改/etc/man.conf,让man调用groff-utf8而不是原来的groff进行格式化操作。

\begin{code}
\#emerge man-pages-zh\_CN groff-utf8
\end{code}

编辑: /etc/man.conf 

找到这一行

{
\tt
NROFF           /usr/bin/groff -mandoc
}

替换为

{ \tt
NROFF           /usr/bin/groff-utf8 -Tutf8 -c -mandoc
}

然后一些汉化了的手册就不会再显示乱码了。
gentoo-zh overlay 已经集成了这一变化,groff-utf8 安装完成后会自动修改/etc/man.conf,如果修改没完成可以手工进行。

\section{符号链接}\label{faq:symlink}

从需求上看,有时候我们希望能使用不同的文件名访问同一个文件。符号链接就是干这个活的。可以理解成一个快捷方式。符合链接分软链接和硬链接两种。硬链接只能用在文件上而无法对目录使用硬链接。而且只有在同一个分区上才能使用硬链接。软链接没有这方面的限制。

硬链接就是在文件系统上出现了使用同一个磁盘区域的多个文件。所以会有上述限制也不奇怪。软链接实际上是个特殊的文件,文件的内容是链接的目标的文件名。打开软链接,实际上会打开链接到的文件。使用readlink(2)系统调用才可以获得链接目标的文件名。

符号链接使用 ln(1) 工具创建。软链接要加“-s”参数。

创建链接的语法是

\begin{code}
ln [-s] 链接到的目标 链接名称
\end{code}


“链接到的目标”可以使用相对路径。如果是相对路径,并不是相对执行命令的时的当前目录,而是相对于“链接名称”的路径。


