\renewcommand{\thesection}{\arabic{section}}

\chapter{FAQ}\label{FAQ}

\section{什么是 LiveCD ?}\label{FAQ:LiveCD}
一种让 Linux 系统脱离硬盘,只需要一张 DVD 或者 CD 运行的技术。使用者只需要插入一张 LiveCD 就可以使用 Linux  系统了。当然,由于 CD-ROM 是只读的,所以对 LiveCD 系统本身所做的更改重启后都会消失。由于 Linux 发行版的膨胀,一张 CD-ROM 无法装下,于是就有了 LiveDVD , 除了容量增加外,和LiveCD 是一致的。此外还有 LiveUSB , 也就是使用 USB 存储设备的 liveCD。用法和 LiveCD 并无二致。

\section{如何使用 LiveCD。}\label{FAQ:UseLiveCD}

\section{UEFI和BIOS是什么?}


\section{什么是API?。}\label{FAQ:API}

\section{Shell 是什么?}
操作系统是由内核和一系列的系统程序组成的。

\section{什么是进程和线程?}\label{FAQ:Process}

\section{什么是内存分页?}\label{FAQ:Paging}

\section{如何不重新登录切换到root帐号?}\label{FAQ:su}

\section{内核参数}\label{FAQ:kernelparamter}



\section{Initramfs是什么} \label{FAQ:initramfs}

Linux启动过程中内核主动做的最后一件事情就是执行/sbin/init,那么倒数第二件事情必然是挂载根目录。
挂载根目录有2个前提条件:找到根分区和支持根分区所使用的文件系统。
找到根分区,第一点,需要其所在设备的驱动。第二点就是内核需要知道哪个是根分区。这个可以通过内核参数 root= 设定。

Linux下驱动可以编译进内核,也可以编译为模块动态加载。如果恰巧磁盘设备的驱动被编译成了模块,就出现“鸡生蛋和蛋生鸡”的问题了:不访问磁盘就无法加载模块,而不加载模块有没有磁盘的驱动。这对于自己编译内核的童鞋不成问题。只要知道自己的SATA/IDE控制器的型号,将对应的驱动编译进内核就解决了。但是如果是打算制作“通用内核”的发行版呢?

文件系统也是同样的问题,没有文件系统驱动就无法挂载分区访问上面的文件,无法访问文件就无法加载模块。这对于自己编译内核的童鞋不成问题。只要将自己根分区所使用的文件系统编译进内核就解决了。但是如果是打算制作“通用内核”的发行版呢?将所有的文件系统都编译进内核么?

initramfs就是为解决这个“鸡生蛋和蛋生鸡”的问题诞生的。

GRUB加载内核的时候通过initrd 命令加载一个initramfs,该initramfs就会成为内核的根分区\footnote{事实上是Linux内核拥有一个不挂载就存在的rootfs文件系统,Linux启动过程中将initramfs的内容解压到rootfs中。作为理解,可以认为initramfs挂载成Linux的临时根目录。}。内核初始化完毕后直接执行initramfs中的/sbin/init而不理会root=命令行参数。由initramfs中的/sbin/init程序负责加载磁盘驱动,查找根分区,挂载根分区到 /sysroot 并调用pivot\_root()\footnote{使用命令 man 2 pivot\_root查看介绍。}将找到的根目录替换为真正的根目录。然后执行真正的/sbin/init程序。

initramfs是一个目录的压缩包。由dracut程序生成。initramfs用于寻找真正的根目录时的临时根目录,通常只包含一些必要的驱动和一些用于查找根分区的脚本,但有时候可以成为一个完成的系统的根目录,这样的系统可以脱离磁盘运行。比如Fedora的安装DVD,使用一个巨大initramfs作为安装环境的根目录。该initramfs包含的是一个真正的OS,带有完整的工具。

