\chapter{黑客和Linux}

\begin{quotes}{Linus Benedict Torvalds}
If 386-BSD had been available when I started on Linux, Linux would probably never had happened.
\end{quotes}

\begin{quotes}{Richard Matthew Stallman}
Writing non-free software is not an ethically legitimate activity, so if people who do this run into trouble, that's good! All businesses based on non-free software ought to fail, and the sooner the better.
\end{quotes}

在计算机发展的初期,软件\footnote{那个时候系统软件都是汇编语言开发的,所以也谈不上什么源代码。}被作为硬件的附属品销售。
购买一个机器的硬件也就连带的附送了一系列的软件。谁也不会觉得软件是可以被“保护起来”盈利的\footnote{机器码可以非常容易的反汇编重新生成源汇编文件。既然软件是汇编开发的,那也没有什么秘密可以隐藏。}。相互交换彼此的程序随处可见。
直到有一天,西雅图的一个软件天才\footnote{为了避免反应迟钝的人不知道,这位天才是比尔\textbullet盖茨啦}说,他花了很多时间写好的软件,不希望被人随意拷贝。每一个获得拷贝的人都应该给他付钱。
有了天才的带领,自此商业公司开始独立销售软件而不再作为硬件的附属品。

不过这股软件私有化的浪潮还不曾传播到社会的各个交流。
大学里,实验室里,人们还是保持着将软件自由传播的传统。虽然他们不再能自由的传播商业软件,但是自己编写的各种
稀奇古怪的软件可以自由的交流。在没有internet的年代他们就开始利用Usenet\footnote{%
\textbf{来自维基百科:}Usenet是一种分布式的互联网交流系统,它的发明是在1979年由杜克大学的研究生Tom Truscott与Jim Ellis所设想出来的。Usenet包含众多新闻组,它是新闻组(异于传统,新闻指交流、信息)及其消息的网络集合。 }相互交流了。
%
这样一群天才不满于商业化的软件,时常有人对商业公司的系统进行破解,故得名hacker,中译“黑客”\footnote{%
由于只有身怀绝技的聪明人物才有可能破解商业软件。故而黑客给人一种智商非常高的印象。久而久之智商很高(或者因为专业技能好,看起来很高)的程序员都被称作黑客。所以现在说的黑客一般指的是计算机方面非常厉害的人而不再是搞破解的人。现在再搞破解和破坏的人,就没有黑客头衔了,而是Cracker,骇客。
}。

Linux的诞生就是因为黑客门不满足于被大公司控制的操作系统市场。

\section{从UNIX到Linux}

早期的UNIX鼓励源代码的密切交流。这和黑客们的思想如出一辙。伯克利开发的BSD更是吸引了大量的黑客进行贡献。70年代到80年代早期,黑客门过着自由的日子。


但是一场突如其来的官司让AT\&T从反托拉斯的禁锢中挣开\footnote{参考前言的内容。}。
脱缰的AT\&T将UNIX推入了深渊。

但是就算UNIX被推入了深渊,BSD还是一个避风港。但是这最后一点希望都随着AT\&T对BSD的起诉而消失了。

1983年,AT\&T发布了Unix最新版system V UNIX,这是一个商业化版本,付费才能使用,并且不得传播源码。
为了减少纠纷,伯克利分校规定,BSD本身依然保持免费,但是只能提供给持有AT\&T源码许可的公司。
不过,与此同时,伯克利的师生也着手开始将AT\&T的代码从BSD中逐渐去除,以便发布一个不受AT\&T控制的版本。
这一举动惹恼了AT\&T。AT\&T起诉BSD侵犯了Unix的版权。
这场诉讼对BSD打击极大,所有的开发活动都被迫停止,用户人心惶惶,担心自己也遭到AT\&T的追究,因此BSD的使用急剧减少。

这场战争给BSD带来了毁灭性的打击。


黑客们突然间没有了玩具。

最后在1994年,双方达到和解,BSD才恢复开发。但是历史的已经不会再给BSD机会了。因为在BSD官司缠身的时候,上帝把爱心悄悄给了Linux。

1991年,芬兰的一个大学生在网上公布了Linux,立即吸引了大量黑客的目光。从此BSD神码的都成了浮云。黑客门团结在天才周围缔造了Linux的神话。
不用说,这位天才的大学生名字就叫做 Linus Benedict Torvalds。

\section{Linux诞生}

UNIX自诞生以来就是学术界研究的对象。UNIX被移植到大量的机器上,唯独没有染指PC市场。UNIX的人轻视了这个以8086为核心构造的个人电脑。
也难怪,UNIX的多任务能力需要CPU能提供一些保护机制,防止未经授权的访问。而这正是8086所缺乏的。

但是当80386发布的时候,情况有了一些变化。80386具备了“真正的”CPU的特性,包括特权级别区分和分页内存支持\footnote{关于分页技术,请参考 \faqref{FAQ:Paging}。}。 有了这些支持,PC终于具备了运行UNIX的条件。

\subsection{被UNIX忽略的PC}
%那么我们还是字为何PC
计算机最初是作为美国军方的项目研究制造的。美军向来有向全世界开放自己的研究的传统\footnote{Internet也源自美国军方的研究。还有现在每天都在用的GPS卫星定位哦~~。},所以计算机得以被商业化。
一家造卡片统计机\footnote{恩,先进的自动化的...算盘}的公司因卡片机的大量销售而积累雄厚的财力和强大的销售服务能力,决定进入未知的电子计算机世界。
并于1944年,开发出自动顺序控制计算机,也被称为Mark  I。

\chatu{ibm_mark1}{Mark I计算机}

随后在1951年它决定开始开发商用电脑,聘请冯·诺依曼担任公司的科学顾问,1952年12月研制出第一台存储程序计算机
\footnote{也就是第一台冯·诺依曼结构的电脑。别和第一台(插拔线路编程的)ENIAC混淆哦!},也是通常意义上的电脑。

没错,这家传奇公司就是IBM。
它开发的第一台电脑就是IBM 701。这是IT历史上一个重要的里程碑。从此计算机进入了程序控制时代,也就是冯·诺依曼结构。

IBM在商用电脑上获得了巨大的成功。那个时候,IBM就是计算机的代名词,有人亲切的称呼IBM为“蓝色巨人”。IBM生产的都是商用电脑,而且都是巨无霸(参考上面的图片),没有一款是适合家庭使用的。
微处理器\footnote{将计算机核心运算和控制部件做到一个芯片上。由于核心非常小,故而得名微型处理器。典型的微型处理器就是Intel为PC生产的80x86系列CPU。}的诞生创造了Apple这样神话。IBM迫切需要一台使用“微处理器”的家用电脑以和Apple这样的新兴公司对抗。

时间紧迫,IBM来不及自己开发操作系统,于是将操作系统外包给了一个西雅图的软件公司\footnote{微软啦!}。CPU也一改自己生产的传统,采购了Intel公司的8088微处理器\footnote{和8086使用相同的指令集,只是在关键的性能指标上比8086次点,因此也有好处:便宜。},最终的结果就是组合出了一台家用微型电脑。俗称PC(Persional Computer 个人电脑)。

PC就是装配着Intel生产的8088/8086处理器,运行着微软的MS-DOS操作系统的微型计算机。

%\subsection{GNU HURD 内核}
但是8086处理器实在赢弱,缺乏实现多任务所必须的功能。所以DOS是个单任务单用户的操作系统。

\begin{insertnote}
\subsection*{小知识:UNIX需要什么样的CPU}
至少要具备2种功能:特权级别和非特权级别划分、支持内存分页(也就是有MMU,Memory Management Unit内存管理单元)。

UNIX将操作系统内核放到特权级别执行,应用程序则使用非特权级别。当处理器处于非特权级别的时候,对内存和IO的访问是严格受限的:访问不允许的IO设备会产生一个异常,该异常被操作系统捕获\footnote{所谓异常捕获,就是处理器发生异常的时候自动跳转到内核提供的异常处理代码。},然后中止试图越权访问的程序;访问不允许访问的内存——如内核使用的内存——会导致内存访问异常,操作系统捕获这个异常,中止试图越权访问的程序。内核自身运行于特权模式。
有了CPU的这种级别划分,UNIX能保护操作系统不受异常程序的影响,提高稳定性和安全性。

UNIX支持多进程运行。每个进程\footnote{参考 \faqref{FAQ:Process}}之间的内存必须进行隔离。也就是说,一个进程只能访问自己的内存,无法访问其他进程的内存。
事实上这个访问不是通过“权限检查”(试图访问其他进程的内存导致进程被中止)实现的,而是通过“完全的隔离”实现的。
完全的隔离,也就是任意进程拥有完整的内存地址空间。并不存在内存地址被划分成N个区域,每个进程只能访问属于自己的$ \frac{1}{N}$的区域。而是每个进程拥有整个的地址。
从 0 开始到 $2^{32}-1$ (对应于32位的系统,如果是64位的则是 $2^{64} - 1$),访问的通通都是属于该进程自身的内存。
不同进程可以使用相同的内存地址访问属于自己的不同的数据。这种内存隔离被称呼为“虚拟内存”。虚拟内存是支持多进程运行必备的功能。

虚拟内存必须依靠CPU的MMU提供的分页功能才能实现。
\end{insertnote}

由于8086/8088处理器并不拥有MMU,也没有实现特权级别划分,所以根本不具备运行UNIX的条件。

就这样PC市场被UNIX忽略,直到80386的出现。80386是Intel生产的第一个具备运行UNIX自个的CPU。
但是就算在功能上具备了运行UNIX的资质,80386从UNIX的观点来说,实在是太慢了。所以谁也没打算将UNIX移植到使用80386处理器的PC上。

PC机就这样继续运行着DOS,以及后来的Windows。

\subsection{386BSD、 Minix催生Linux}

要说PC真的没有一个UNIX可以运行那是假的。商业UNIX抛弃了PC,但是黑客们却像发现了新大陆一样为PC着迷。于是,各种黑客版本的UNIX被移植到80386芯片上。
其中最著名的莫过于386BSD(将BSD移植到80386处理器的项目)和Andrew S. Tanenbaum教授以教学为目的开发的Minix。

Linus在大学里用过UNIX就为它着迷。于是决定为自己买一台计算机运行UNIX,这样就不要跑去学校的机房了。
由于PC性能不佳,跑着UNIX的工作站又太贵了。Linus决定购买一台PC。时值BSD正好处于和AT\&T的长期诉讼中。法律风险是大大的。于是Linus装起了Minix。
至于其他版本的UNIX,就没入Linus的法眼,呵呵。

Minix是Andrew S. Tanenbaum教授以教学为目的开发的UNIX,但是并不是一个“自由”的软件。而且Linus对于Minix有相当多的吐槽。不仅仅是Linus,许多人都为他们的吐槽写了补丁。
但是Andrew以“只用于教学目的的系统不想变的太复杂”为由而拒绝添加这些改进。一个拒绝改进的开发者+虽然宽松但是不自由的授权\footnote{
购买Andrew的《操作系统设计与实现》这本书就能获得Minix代码并授权个人使用。仅限于非商业用途。}\textrightarrow ~Linus~决定\textbf{为自己}开发一个UNIX兼容的操作系统。


\subsection{以GPL开源}

Linus说自己在Linux上做的最正确的决定就是用GPL保护了Linux。1911年发布的第一版内核并不是使用GPL协议发布的。
他使用了自定的一个协议,内容大概就是:{\it Linux自由免费,随便用。如果你对Linux做了修改,麻烦你把修改后的代码给我一份,谢谢。}

在黑客界,有一个古老的传统:\emph{尽量去发现工具,实在没有就自己写一个。} 这条传统似乎对软件授权协议也有效。
%
Linus大概觉得自己跳过了第一步的发现,于是他决定重新开始依照老传统做一次——这次他发现了GPL,于是他不再发明,将Linux置于GPL的保护之下。

Linus为第一版Linux编写的协议和GPL想要表达的意思差不多,但是起草过程中有律师帮忙的GPL更加的严谨。

从头开发的Linux源码和GPL授权为Linux的发展构建了坚实的后盾。即没有侵权的后顾之忧也没有不怕被侵权后缺乏法律的保护。Linux就这样替代了386BSD,Minix等等等等
成为了PC上最受欢迎的类UNIX系统。同时由于Linux使用GPL授权,填补了GNU缺失的内核 GNU Hurd。

全世界的黑客开始行动起来,推动着Linux快速的发展。Linux成了分布式开发的典范。
