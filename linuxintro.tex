
Linus发布他的内核的时候,Linux就只是一个内核而已。一个操作系统不当当依靠一个内核就能工作。还需要各种程序供用户使用。最少的,用户需要一个shell进行人机交互。
用户需要管理磁盘,管理他的文件。等等操作都需要依靠各种各样的程序进行。如果用户只是将Linux内核运行起来,那他将什么都做不了。所以用户需要的,是一个Linux内核和一套工具软件。这样的组合才能构成用户操作机器的环境——也就是说,一个操作系统。于是人们有时候简单的称呼这样的操作系统为Linux。但是我们得知道,这只是泛称,Linux的实际含义就是Linus和其他黑客开发的一个内核,并不是一个完整的操作系统。

由于Linux只是一个内核,所以用户需要的工具软件就需要向别的地方寻找了。所幸RMS宣誓的GNU已经开发的差不多了,就只差个名字叫GNU Hurd的内核。等等,我们不是有Linux内核了么?

没错!只要我们将GNU的系列软件和Linux内核搭配起来,不就有一个完整的操作系统了么?我们需要一个shell, GNU bash就是我们要找的。我们需要各种命令行工具, GNU coreutils就是我们要找的,我们需要编辑器,GNU emacs就行。我们需要开发工具编写自己的程序,GCC GDB GNU make等等就是我们需要的。 我们要的各种构成UNIX系统的要素GNU通通提供了。额,除了内核。所以RMS提议使用Linux内核+GNU userland\footnote{
前文提到过,UNIX需要CPU提供特权和非特权两个代码级别。内核的代码运行在特权级别。用户的程序运行在非特权级别。故而用户程序统称userland。}的操作系统应该叫做GNU/Linux。

那还等什么,赶紧到GNU的网站上下载吧!

\subsection{使用Linux内核的操作系统——发行版}
且慢!我还未拥有一个UNIX环境,GNU网站上提供的都是源代码,Linux也是源代码,我如何使用他们?编译?
那么我首先需要一个UNIX系统,并有一个能工作的编译器,才能编译这些软件吧!

这是一个先有鸡还是先有蛋的问题。

再说,就算我已经安装了一个Minix或者BSD系统,想编译出一个使用Linux内核的完整操作系统,也是一个巨大的工程吧!

俗话说,有需求的地方就会有市场。

于是有那么些人开始将Linux内核以及各种软件搜集到一起,预先从源码编译成二进制的可执行程序然后打包放到光盘上出售。并编写了方便的安装程序。只要把光盘塞入PC的光驱,跟着光盘里安装程序的提示一步一步做下去就能安装好一个完整的系统。

这种以Linux为内核的操作系统和其他的软件一起打包组合而成的套装就是发行版。


\section{发行版的意义}
当然,发行版的意义并不是简单的将软件打包和降低初学者的门槛哦\textasciitilde

第一:通过将大量的程序打包到统一的仓库,使得很多默默无闻的软件被显示到可安装软件列表上。增加了这些低调的软件开发者的潜在用户。虽然说用户多了开发者不一定有好处,但是光是想到有那么多人会用自己开发的软件,想想都觉得来劲不是么?

第二:许多软件特定版本的\emph{组合}可能会导致一些潜在bug。这些是软件开发者无从测试的。一个特定版本的发行版将基础软件的版本全部确定下来。只要保证这些版本的配合不会产生bug就可以了。由于用户使用发行版提供的软件,就减少了各种没有预料到的组合。所以在生产环境,随机的升级某个软件是非常忌讳的哦\textasciitilde。

第三:许多软件开发商形容Linux是活动的靶子。每过6个月\footnote{Ubuntu每6个月发行新的版本。},系统各个软件都版本大换血。原先测试好的软件活不过6个月——下一个版本出来就出各种问题了。但是开源软件因为是发行版负责打包编译,所以可以避免这儿个问题。发行版维护者会及时发行某些软件因为另一些软件的升级而不工作了,通常简单的问题维护者自己就可以将错误修正——同时将修改回馈上游。开源软件可以放心的躲过每6个月一次的劫数——除了拒绝开源的。

第四:第三条提过,发行版积极的向上游回馈修改\footnote{咳咳,并不是每个发行版都那么好心。Ubuntu就曾经因为回馈不足引发舌战过。}。
有时候发行版回馈的补丁不仅是躲过6月劫数那么简单。一些发行版直接参与上游软件的开发,特别是和发行版结合紧密的软件。Fedora是RedHat赞助的社区发行版,默认使用GNOME桌面,而GNOME有相当一部分开发者就是RedHat雇员。


\section{包管理和软件仓库}

是什么东西促使一堆软件的集合变成了一个发行版呢?答案是包管理机制和围绕它建立的一个软件仓库。

\subsection{包管理}

人要用电脑,就免不了要安装软件。浏览器,办公软件,游戏等等。软件安装了,免不了要升级和卸载。
%人发行版需要

所有的程序都带上升级卸载程序,是个不错的主意,也是个愚蠢的主意。不符合UNIX的KISS原则。

符合KISS原则:
\begin{itemize}
\item 安装程序应该只有一份。
\item 所有的软件使用统一的方式安装,不能区别对待

\item 使用方便。提供直观间接的命令行或者图形界面。

\item 干净卸载。卸载不能留下任何不该留下的东西。所谓燕过不留痕。

\item 解决依赖问题。一个软件或多或少的会依赖其他的软件。最明显的例子,如果系统里没有安装libc,所有的软件都不可能运行。一个软件要想正常工作,它依赖的软件必定也要安装。 包管理要最终软件之间的相互依赖关系。安装一个软件的时候能将它依赖的软件也安装,卸载软件的时候能将依赖的软件(如果没有其他的软件也依赖它的话)一并删除。

\item 打包方便,不能方便了最终用户而是发行版维护人员痛苦不堪。

\end{itemize}

各个发行版使出浑身解数,开发了各种软件包格式和对于包管理器。比较流行的也就只有两个而已:RPM和deb。

\subsection*{RPM}

RPM是Redhat Package Manager的缩写,不过后来RPM被Redhat以外的发行版采用的,似乎RPM成了另一种缩写:Rpm Package Manager\footnote{吐槽吧,学的GNU啊。递归缩写}。

RPM格式的扩展名是.rpm,分源码包和二进制包。源码包将rpm规格说明文件(.spec)和程序源码打包为一个.src.rpm包(或者.srpm) 。源码包可以用来构建二进制rpm包。
要安装rpm包,使用rpm命令进行。具体使用请参考man手册。比较常规的用法我会在
%\secref{软件管理} 介绍。
第\ref{软件管理}章~\nameref{软件管理}介绍。

\begin{insertnote}
\vskip 0.5cm
man手册的用法很简单啦,打man后面跟个要帮助的命令,rpm就是
\begin{code}
\$man rpm
\end{code}
按q退出查看。PgUp和PgDown可以翻页。
\end{insertnote}

RPM被广泛使用,包括RHEL、Fedora、OpenSuSE、RedFlag都是使用的rpm包格式。

\subsection*{deb}

包管理世界的另一大流派是deb。deb因其所使用的软件包扩展名为.deb而得名。实质上是Debian的缩写。因为它是Debian专用软件包格式。
因为Ubuntu是Debian修改而来,自然也采用了deb包管理。

和RPM一样,deb也有源码包和二进制包。从源码包可以编译得到二进制包。要安装deb包,使用dpkg命令进行。具体使用请参考man手册。比较常规的用法我会在第\ref{chap:软件管理}章~\nameref{chap:软件管理}介绍。

\subsection*{ebuild}

Gentoo的ebuild严格来说并不是软件包格式。rpm deb之类的,可以直接安装对应的二进制包。包管理器检查包的依赖,依赖检查后将包里的文件解压安装到系统里就可以了。

Gentoo的ebuild却并不是这样的。ebuild大致是一个BASH脚本(不能直接执行,只能由包管理器调用)。Gentoo的包管理器通过执行ebuild完成“自动下载源码,解压,编译,安装”这样的操作。

%TODO
\todo{更多ebuild介绍}

\subsection{软件仓库}

发行版搜集了各种软件,并编译成二进制包。通常发行版收集的软件包数量以万计。这么多包是不可能一次性全部装到一张光盘上的。也完全没有那种必要。用户拿到光盘的那一刻开始软件就已经过时了。

如果将所有的包都集中存放到服务器上,用户要安装的时候直接到服务器上获取就可以了。用户获得的软件总能保证是最新的。只需要比较本地安装的软件版本和服务器上存储的最新版本就可以知道是否有更新了。

这样在服务器上集中存储的软件包就集合成了一个仓库。发行版的工作就是维护这个仓库,时不时的向仓库更新软件。

软件仓库可是个伟大的发明。乔帮主从Linux社区学到了软件仓库创立了App Store赚了大发。微软也开始学软件仓库的思想,Windows 8也开设了应用程序商店。

限于仓库的格式问题,不同的发行版有不同的工具让用户访问软件仓库。

RPM系的发行版,使用yum\footnote{不要和rpm混淆哦!一个是安装软件包的,一个是访问软件仓库的。}访问仓库。用户可以下载 XXX.rpm 然后使用 rpm -i XXX.rpm 安装这个包,但是更便捷的做法是直接使用 {\tt yum install XXX} 安装名为XXX的软件。yum会到仓库里搜索XXX软件并将对应的rpm文件下载下来,如果有未安装的依赖,将依赖的软件也下载下来。然后调用rpm安装。

\begin{example}{Fedora下安装手写输入法}
手写,handwrite,搜索
\begin{code}
\#yum search handwrite
\end{code}

在结果页面,有个叫 ibus-handwrite的软件。ibus是一个输入法框架,自然ibus-handwrite就是手写输入法了。
开始安装(注意需要使用root帐号运行,参考\faqref{FAQ:su})
\begin{code}
\#yum install ibus-handwrite
\end{code}
回答yes后ibus-handwrite就被安装到系统上了。
当然,前提是保证网络连接:)
\end{example}

要删除软件,使用 yum remove XXX.

对deb系的发行版,使用apt-get(俗称超级牛力)哦。基本用法和yum一致。apt-get install 安装软件,apt-get remove 卸载软件。
区别就是,yum是在线搜索,apt-get是离线搜索。你需要每天使用apt-get update(也不需要每天啦!
长久没用apg-get的时候先执行一下apt-get update就行了。)刷新缓存,
之后apt-get就使用本地缓存的软件包信息而不用每次都连接到服务器进行搜索。

另外,更新系统yum使用的操作是 yum update,而apt-get同样的操作用来更新本地软件包信息缓存了,故而更新系统在apt-get下使用的命令是apt-get upgrade。

而Gentoo这样的系统嘛,也有非常强大的工具:{\bf emerge}。emerge XXX就可以安装名为XXX的软件。

\begin{example}{Gentoo下安装vim}
要安装vim非常简单,emerge就可以了
\begin{code}
\#emerge vim
\end{code}
emerge会忠实的到vim的网站上下载vim的代码,解压,编译,然后安装。如果vim依赖的软件还未被安装的话,emerge会首先安装它们。

\end{example}

\subsection*{在仓库中搜索}

虽然仓库很好用,但是在用的时候大家都会有一个疑问:怎么知道一个软件对应的包的名字?

这就需要到仓库中搜索需要的软件了。搜索条件一般为软件的名字。通常软件包的名字和软件本身的名字一致。但也有例外。就拿glibc来说,在deb系里,它的包名是libc6。
这些问题都有历史渊源,说来话长,咱就不废话了。总之记住:软件包的名字并不总是和软件本身的名字一致。所以有时候需要搜索后才能安装。另外一个问题是,有时候我们只是希望安装一个“类型”的软件,比如想安装个画图软件,但是恰巧并不知道任何一款画图软件的名字。

针对这2个问题有两个办法。

第一个问题,属于特定发行版下的“软件包改名”问题。一般使用发行版的软件管理工具的搜索命令。以软件本身的名字作为关键词搜索即可。比如 yum search vim、apt-cache search sudo、emerge -s vnc。

第二个问题,命令行下无解。如果使用图形工具可以方便的浏览仓库中所有的软件。还是分门别类的哦!当然,最快的办法莫过于祭出伟大的Google。
